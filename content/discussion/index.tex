\section{Discussion}
\note{
    \begin{itemize}
        \item Notes...
    \end{itemize}
}

\metroset{block=transparent}

\subsection{Alternative Delay Functions}

\begin{frame}{Alternative Delay Functions}
    \centering
    A dedicated ASIC to compute \emph{sloth} may be several\\orders of magnitudes faster.
    \\
    \vspace{1cm}
    Adversary with ASIC must also execute an attack.
\end{frame}
\note{
    \begin{itemize}
        \item Fast enough to pose a problem
        \item (Because the asic is so fast, a user might not be able to see a commitment fast enough to trust it --- because network latency might be relatively high)
        \item However, adversary must also be able to successfully execute an attack. The ASIC alone is not enough.
        \item Still, we want to deter the usage of ASICs.
        \item How can we do that?
    \end{itemize} 
}

\begin{frame}{Alternative Delay Functions}
    \centering
    \begin{columns}[T, onlytextwidth]
        \column{0.47\textwidth}
        \begin{block}{Use Memory-Hard Delay Function}
            \begin{itemize}
                \item ASICs have limited memory.
                \item Requiring e.g.\ 1 gb of memory per delay function would make ASICs infeasible
            \end{itemize}
        \end{block}

        \column{0.47\textwidth}
        \begin{block}{Change Delay Function}
            \begin{itemize}
                \item An ASIC costs a lot and is hard-wired
                \item Changing the delay function is free
                \item We can even just permutate the delay function.
            \end{itemize}
        \end{block}
    \end{columns}
\end{frame}
\note{
    \begin{itemize}
        \item Memory-hard: 
        \begin{itemize}
            \item 1 gb infeasible: That amount of memory can only be slow general RAM. Not fast enough for ASIC.
        \end{itemize}
        \item Change/Permutate:
        \begin{itemize}
            \item asdfasdlf
        \end{itemize}
    \end{itemize}
}


\subsection{Unmitigated Threats}

\begin{frame}{Unmitigated Threats}
    \begin{columns}[T,onlytextwidth]
        \column{0.47\textwidth}

        \begin{block}{Input Flooding}
          \begin{itemize}
              \item Rate limiting
              \item Proof-of-Work puzzle
          \end{itemize}
        \end{block}

        \begin{block}{Eclipse Attacks}
            \begin{itemize}
                \item We accept carrier pidgeon?
            \end{itemize}
        \end{block}

        \column{0.47\textwidth}

        \begin{block}{Operator Shutdown}
            \begin{itemize}
                \item Eventually will drive users away
            \end{itemize}
        \end{block}

        \begin{block}{Cryptography Exploit}
            \begin{itemize}
                \item We cannot anticipate these attacks
                \item We are not dependent on any algorithm
            \end{itemize}
        \end{block}

      \end{columns}
\end{frame}
\note{
    Notes...
}

\subsection{Practicalities}

\begin{frame}{Practicalities: Stopping in a Fair Way}
    A common necessity for most real world applications that requires longer input collection time than e.g.\ 1 minute.

    Stop message needed to make the stopping entity unable to stop when they deem it most beneficial. They can stop, and then see the result (commitment).

    Pattern, usability apps can support this in a standardized way (we cannot assume every application can get it right anyways).
\end{frame}
\note{
    Notes...
}

\begin{frame}{Practicalities: Extending Trust into the Reality}
    How can we ensure that the lottery pays the money?
    \begin{itemize}
        \item Trust them and use third-party entity in case of problems (e.g.\ court).
        \item Cryptocurrencies and smart contracts.
    \end{itemize}
\end{frame}
\note{
    Notes...
}
