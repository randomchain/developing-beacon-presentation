\section{Architecture}
\note{
    \begin{itemize}
        \item yay
    \end{itemize}
}

\begin{frame}[t]{Architecture}
    \textbf{Service Oriented}
    \begin{itemize}
        \item Separation of concerns
        \item Components/Services are black boxes
        \item Must adhere to predefined communication API
        \item Components can be added or removed dynamically
        \item Scalability and fault tolerance
        \item Compose a beacon to fit needs
    \end{itemize}
    \note{
        \begin{itemize}
            \item Uhm.
        \end{itemize}
    }
\end{frame}
\begin{frame}[t]{Architecture}
    \textbf{Pipeline}
    \begin{itemize}
        \item More strict pattern
        \item Data flows in one direction
        \item Components are producers, consumers, or both
        \item No cyclic communication
        \item Clean interface
    \end{itemize}
    \note{
        \begin{itemize}
            \item Generally we are not concerned with ``lost'' data
            \item This motivates non-fragile use, i.e.\ users should not rely on the beacon in a critical manner

            \item We send data, not commands --- no state or side-effects (functional)
        \end{itemize}
    }
\end{frame}

\begin{frame}{Implementation}
    \makebox[\linewidth][c]{%
        \begin{minipage}{\dimexpr\textwidth+4.5em\relax}
            \raggedright%
            \centering
            \subimport{}{impl_figure.tex}
        \end{minipage}%
    }
\note<1>{\textbf{Start with input processor.}
    \begin{itemize}
        \item Uses tmux to separate the beacon into tree windows.
        \item Inputs are aggregated
        \item When ready to compute merkle tree is built.
        \item Work is sent to available workers.
        \item But we are not able to receive user inputs yet.
    \end{itemize}
}
\note<2>{\textbf{Next up is input collection.}
    \begin{itemize}
        \item Proxy to allow for easier connection etc.
        \item Stream proxy is fan-in/fan-out --- fair message distribution.
        \item Proxy allows for multiple input ``aggregators'' which e.g.\ can build sub merkle trees.
        \item Three input collectors.
    \end{itemize}
}
\note<3>{\textbf{We need to do some computing to get outputs}
    \begin{itemize}
        \item In this case three computational nodes are deployed.
        \item Announces themselves to input processor, and receive work as per the schedule.
    \end{itemize}
}
\note<4>{\textbf{Finally, we must publish outputs, commitments, and proofs.}
    \begin{itemize}
        \item Also using a proxy here.
        \item Two publishers, note that twitter only publishes proofs! --- ``Gimmick''
    \end{itemize}
}
\end{frame}

\section{Usability}

\begin{frame}[t]{User Interaction}
    \begin{itemize}
        \item Beacon publishes to outlets: files, Twitter, etc.
        \item Users want convenience --- using the beacon should be painless
        \item 3rd party usability applications
    \end{itemize}
\end{frame}
\note{
    \begin{itemize}
        \item We need to be aware of the user interaction with our beacon
        \item The beacon itself should be kept as simple as possible
        \item Features often means complexity and thereby bugs
        \item Users should not need to interact directly with the beacon --- But it should be possible to do so! \textbf{Transparency}
    \end{itemize}
}

\newlength{\smallerspacing}
\setlength{\smallerspacing}{-.5em}

\begin{frame}[t]{Usability Applications}
    \note{
    \begin{itemize}
        \item CLI demo
        \item Web App demo
        \item Other possibilities
        \item Daemon can remove ``pain'' in using beacon
        \item API can facilitate custom usage, e.g.\ lotteries, protocol bootstrap
        \item API could also be for smart contracts!
    \end{itemize}
}\begin{columns}[T,onlytextwidth]
        \column{0.49\textwidth}

        \begin{block}{Verification CLI}
            \small
            \vspace{-.5em}
            \begin{itemize}
                \item Check inclusion of an input\itemsep=\smallerspacing
                \item Validate in Merkle tree\itemsep=\smallerspacing
                \item Verify computation (sloth)
            \end{itemize}
            \vspace{-.5em}
        \end{block}

        \only<3->{%
        \begin{block}{User Daemon}
            \small
            \vspace{-.5em}
            \begin{itemize}
                \item Constantly input in background\itemsep=\smallerspacing
                \item Check all commitments\itemsep=\smallerspacing
                \item Verify all outputs\itemsep=\smallerspacing
                \item Supply trustable outputs to user
            \end{itemize}
            \vspace{-.5em}
        \end{block}
        }

        \pause

        \column{0.49\textwidth}

        \begin{block}{Web App}
            \small
            \vspace{-.5em}
            \begin{itemize}
                \item Give input\itemsep=\smallerspacing
                \item Check commits\itemsep=\smallerspacing
                \item Explore Merkle tree\itemsep=\smallerspacing
                \item Explore output and proof
            \end{itemize}
            \vspace{-.5em}
        \end{block}

        \only<3->{%
        \begin{block}{API}
            \small
            \vspace{-.5em}
            \begin{itemize}
                \item Use beacon in the background\itemsep=\smallerspacing
                \item Present trustable randomness\itemsep=\smallerspacing
                \item Used to implement ceremonies
            \end{itemize}
            \vspace{-.5em}
        \end{block}
        }
    \end{columns}
\end{frame}


% SOA/Pipeline
%   Implications
%   Pros/cons
% Look at the architecture figure
% Relate to actual implementation
%   Each component and how it fits into the figure
%   How we communicate
% Spin up a new beacon
%   Go from 1 to multiple input collectors
%   Add computation nodes dynamically (maybe?)
%   Add publishers


% Quickly demonstrate possible usability apps
%   Verification CLI
%   Web app ??
% Present other possible supporting apps
